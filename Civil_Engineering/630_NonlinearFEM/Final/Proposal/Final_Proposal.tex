\documentclass[12pt,a4paper]{article}
\usepackage[utf8]{inputenc}
\usepackage{amsmath}
\usepackage{amsfonts}
\usepackage{amssymb}
\usepackage{graphicx}
\author{Carter Rhea}
\title{Non-linear Finite Element Method Final Project Proposal}
\begin{document}
\maketitle

\section{Introduction}
MOOSE, a massively parallel, adaptive finite element platform developed at Idaho National Laboratory, allows users to develop code to handle nonlinear systems. I propose to use MOOSE to study the evolution of  fracture in nonlinear elastic systems. Our lab has been developing a continuum model to examine fracture in linearly elastic systems using the phase field modeling scheme. We have one model which simply looks at cracks due to mode 1 and mode 2 fracture, while we also simulate fluid-driven fracture. Our current simulations take advantage of phase field modeling to follow crack propagation. I propose to continue using phase field for fracture evolution; However, in contrast to our current model, I would decouple the damage field (phase field variable for crack growth) from the stress field, thus using it purely as a concentration field of the fluid and to give rise to the pressure in the cracked region. \\
\section{Objectives}
\begin{enumerate}
\item Implement in MOOSE an arc-length solver for systems of multiple degrees of freedom (x displacement, y displacement, pressure).
\item Use arclength solver to handle problems in which the constitutive law exhibits softening
\item Enhance current code to investigate the effects of nonlinear strain on 2d surfaces undergoing uniaxial tension. Namely, a bar with differing open-geometries in the center.

\end{enumerate}

\end{document}